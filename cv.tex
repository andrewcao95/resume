
%  ______     __     __  ____                                
%  / ___\ \   / /    / / |  _ \ ___  ___ _   _ _ __ ___   ___ 
% | |    \ \ / /    / /  | |_) / _ \/ __| | | | '_ ` _ \ / _ \
% | |___  \ V /    / /   |  _ <  __/\__ \ |_| | | | | | |  __/
%  \____|  \_/    /_/    |_| \_\___||___/\__,_|_| |_| |_|\___|
%

% My personal resume...

\documentclass{resume}

\begin{document}

\name{Amlesh Sivanantham}
\contact{3025 Royal Street, Los Angeles, CA 90007}
{samlesh@gmail.com}{sivanant@usc.edu}{(408) 219-6474}
{zamlz}{https://github.com/zamlz}

% Comment out skills that are not relevent to the job application
\imgsection{skills}{Skills}
\skillsubsection{Languages}{
Python,
C, 
C++,
\LaTeX,
Bash,
Zsh,
Javascript,
HTML
}
\skillsubsection{Libraries}{
TensorFlow,
NumPy,
OpenCV
}
\skillsubsection{Technologies}{
Linux, 
Unix,
Git,
Scrum,
Vim
}

\imgsection{education}{Education}
\datedsubsection{University of Southern California}{August 2017 - Present}
Computer Science: Intelligent Robotics M.S.

\datedsubsection{University of California, Santa Cruz}{September 2013 - June 2017}
Computer Engineering B.S. \\ Computer Science B.S.

% Like the skills section, comment out irrelevent projects
\imgsection{projects}{Projects}

\projectsubsection{Quadcopter Reinforcement Learning Agent}
{USC Robotics Embedded Systems Laboratory}
{Working on building an Reinforcement Learning agent that is capable of
navigating quickly through a cluttered environment. Train the agent in
simulation, and transfer the learnt policy to a real-world quadcopter.
\textit{IN PROGRESS}}

\projectsubsection{Anomaly Detection in Time Series Data}
{Undergraduate Senior Thesis}
{Researched Deep Learning and implemented a long short-term memory
network that identifies if a given time-series sequence of a particular
time series system is anomalous or not. The dataset that I worked with
was provided by my faculty advisor which corresponds to the energy usage
of an electric meter on the circuit that provides power to the water
pump. After training, the network was able to identify anomalies with
an accuracy of 90\%.}

\projectsubsectionold{Who's Lazy? Not Eye}
{Hack UCSC 2017}
{A vision therapy program for people with lazy eye using a standard 
webcam. The app uses the webcam to constantly analyze the user's eyes 
and notifies them when their eyes drift away. Particularly, the client
can pause any media application that the user has playing in the 
background and will only let them resume their application once they 
have focused their eyes. I worked on the algorithm that located the
position of the pupils using machine learning and identified whether
the pupils correlated with lazy eyes or not based on that position.}

% Like with all the other sections, comments the ones you don't need
\imgsection{work}{Experience}

\jobsubsection{Graduate Research Assistant}{September 2017 - Present}
{University of Southern California - Robotic Embedded Systems Laboratory}
{Perform graduate research in Deep Reinforcement Learning and it's application
to Robotics. Work under a PhD student on a project.}

\jobsubsection{Undergraduate Research Assistant}{September 2016 - June 2017}
{University of California, Santa Cruz - Jack Baskin School of Engineering}
{Performed undergraduate research in Machine Learning and Deep Learning 
for the S.E.A.D.S. project to study ways to analyze time-series data.}

\jobsubsectionold{Class Grader}{April 2017 - June 2017}
{University of California, Santa Cruz - Jack Baskin School of Engineering}
{Graded homework for Computational Models (CMPS 130) and Analysis of 
Algorithms (CMPS 102).}

\imgsection{hobbies}{Research Interests and Hobbies}
I am interested in Artificial Intelligence and Robotics. Particularly,
I want to focus on the field of Deep Learning and its application to
cyberphysical and social problems. I am also very interested in 
Deep Reinforcement Learning such as AlphaGo. My other hobbies include a variety
of things such as researching about Linux, Physics and Astronomy. I
also practice the Violin and can solve a variety of different rubik's
cubes.

\end{document}
                                                            
